\documentclass[a4paper,10pt]{article}
\usepackage[french]{babel}
\usepackage[utf8]{inputenc}
\usepackage[left=2.5cm,top=2cm,right=2.5cm,nohead,nofoot]{geometry}
\usepackage{url}
\usepackage{graphicx}
\usepackage{hyperref}

\linespread{1.1}



\begin{document}

\begin{titlepage}
\begin{center}
\textbf{\textsc{UNIVERSIT\'E LIBRE DE BRUXELLES}}\\
%\textbf{\textsc{Faculté des Sciences}}\\
%\textbf{\textsc{Département d'Informatique}}
\vfill{}\vfill{}
\begin{center}{\Huge Rapport : HORECA}\end{center}{\Huge \par}
\begin{center}{\large Thomas Perale}\end{center}{\Huge \par}
\vfill{}\vfill{} \vfill{}
\begin{flushleft}{\large \textbf{INFO-H-303 Base de données}}\hfill{Esteban Zimányi, Michaël Waumans}\end{flushleft}{\large\par}
\vfill{}\vfill{}\enlargethispage{3cm}
\textbf{Année académique 2015-2016}
\end{center}
\end{titlepage}

%\begin{abstract}
%Ce rapport présente ...
%\end{abstract}


\tableofcontents

\pagebreak


\section{Diagramme entité association}
\subsection{Diagramme}
\begin{figure}[hbt]
  \includegraphics[scale=0.4]{bdd.png}
  \caption{Diagramme entité association
}
\end{figure}
\subsection{Contraintes}
Les contraintes sont les suivantes :
\begin{itemize}
  \item Le couple (Longitude, Latitude) est unique.
  \item La longitude et la latitude doivent être comris entre -180 et 180.
  \item Le nom d'utilisateur est unique.
  \item L'email d'utilisateur est unique.
  \item Pour les commentaire, le couple (IdUtilisateur, date) est unique. Donc
      un même utilisateur ne peut pas écrire deux commentaires en même temps.
  \item Pour les labels, le triplet : (Label, IdArticle, IdUtilisateur) est
      unique, ça veut dire qu'une personne ne peut pas ajouter deux fois le
      même label.
  \item Le nombre d'étoile (commentaire et hotel), doit être compris entre 1 et 5.
  \item Les ID des hotels/bars/restaurants référencent ceux de la table
      établissement, chaque établissement est soit un hotel, soit un bar, soit
      un restaurant, il ne peut pas être deux type d'établissement en même
      temps.
  \item La Date de création d'un établissement doit précéder celle de ses commentaires.
  \item La Date d'enregistrement d'un admin doit précéder celle de création d'un établissement par celui-ci.
  \item La Date d'enregistrement d'un utilisateur doit être antérieur à celle où il a écrit un commentaire.
\end{itemize}



\section{Modèle relationnel}
\subsection{Modèle}

\begin{description}
\item[] \textbf{Utilisateur}(\underline{Id}, Email, MotDePasse, DateEnregistrement, Admin(0, 1))
    \begin{description}
        \item[] Utilisateur.Admin indique si l'utilisateur est un admin ou non.
    \end{description}

\item[] \textbf{Etablissement}(\underline{Id}, Nom, Adresse.Rue,
    Adresse.Numéro, Adresse.CodePostal, Adresse.Localité, Coordonnée.Latitude,
    Coordonnée.Longitude, Téléphone, Site, DateCreation, Creator)
    \begin{description}
        \item[] Etablissement.Creator réfèrence Utilisateur.Id.
    \end{description}

\item[] \textbf{Restaurant}(\underline{Id}, Prix, Places, AEmporter, Livraison, horaire)
    \begin{description}
        \item[] Restaurant.Id réfèrence Etablissement.Id.
    \end{description}

\item[] \textbf{Bar}(\underline{Id}, Fumeur, Snacks)
    \begin{description}
        \item[] Bar.Id réfèrence Etablissement.Id.
    \end{description}

\item[] \textbf{Hotel}(\underline{Id}, NombreEtoile, NombreChambre, Prix)
    \begin{description}
        \item[] Hotel.Id réfèrence Etablissement.Id.
    \end{description}

\item[] \textbf{Commentaire}(\underline{Id}, Commentaire, NombreEtoile, Date,
    Image, IdArticle, IdUtilisateur)
    \begin{description}
        \item[] Commentaire.IdArticle réfèrence Etablissement.Id.
        \item[] Commentaire.IdUtilisateur réfèrence Utilisateur.Id.
    \end{description}

\item[] \textbf{Label}(\underline{Id}, Label, IdArticle, IdUtilisateur)
    \begin{description}
        \item[] Commentaire.IdArticle réfèrence Etablissement.Id.
        \item[] Commentaire.IdUtilisateur réfèrence Utilisateur.Id.
    \end{description}
\end{description}

\subsection{Contraintes}

\begin{itemize}
    \item On doit d'abord créer l'établissement avant de créer sa
        spécialisation (restaurant, bar, hotel) de manière à pouvoir
        référencer cet établissement.
\end{itemize}

\section{Hypothèses}
Les utilisateur "admin" sont encodé directement dans la base de donnée.
\newline

Il est marqué dans l'énoncé :
\begin{quote}
    Les utilisateurs peuvent commenter plusieurs fois le même
    établissement à des dates différentes.
\end{quote}
Les dates seront stocké dans ma base de donnée sous forme de \emph{timestamp},
dés lors un utilisateur ne pourra pas envoyer deux message lors de la même seconde.
\newline

Un administrateur peut modifier un établissement, mais j'ai décidé de ne pas
stocker les dates de modification mais seulement celle de création.
\newline

Les adresses mails ainsi que les mot de passes des \emph{Utilisateur} qui nous
sont fournis dans le fichier \emph{Data.zip} peuvent être choisis.

\section{Justification}

J'ai décidé pour la généralisation des établissements d'hériter les clées
principales qui sont dans établissement. De cette manière j'évite de devoir
redéclarer à chaque fois dans chaques table spécialisée (hotel, restaurant ou
bar) les mêmes données. De plus la recherche de d'établissement par nom est
rendue plus simple. \newline

Pour le généralistion des utilisateurs j'ai décidé d'ajouter une colonne
"Admin" dans la table utilisateur pour permettre de savoir si un utilisateur
est admin ou non. \newline

En ce qui concerne les \emph{labels} j'ai décidé de mettre l'\emph{ID} de
l'utilisateur dans une des colonnes, de cette façon on sait facilement savoir
si une personne a déjà ajouté un certain label ou non.

\end{document}
